Tapered microstrip balun for ATA feed development
G. Engargiola

Abstract:

Introduction:

The current design of the ATA feed is a non-planar
log-periodic antenna with a 20 degree opening angle. Between the
arms of the antenna is located a square pyramidal metallic shield
with a 10 degree opening angle, which serves as a vacuum container 
for a cryogenic amplifier. The self-similar geometry of
the antenna/shield combination produces a radiation pattern and
impedance which vary insignificantly over a log period in
frequency. A connection from the balanced leads of the antenna
to the electronics interior to the shield can be made in two ways:
Each antenna arm can be connected via an unbalanced transmission
line to an independant unbalanced amplifier input. Opposite antenna
arms produce antisymmetric signals, hence their amplified outputs
can be combined with a 180 degree hybrid; a scheme for this connection
is described in a separate memo. Alternatively, the balanced terminals
of the antenna can be connected via a balun to a single amplifier.
The balun must be appropriately broadband, matching the terminal
impedance of the antenna to the amplifier input, and compatible with
the long narrow geometry of the shield at the location of the
antenna terminals. I present a design for such a balun in this memo.

Basic design:

Simply put, a balun is a reciprocal transducer, converting the odd signal
mode at the antenna terminals to the sum of an odd and even mode 
at the terminals of an unbalanced (grounded) transmission line structure.
In effect, it passes microwave energy while blocking net rf current
flow between the two ports. Space contraints in the pyramidal
shield and the need for wide bandwidth make a taperline design ideal --
ferrite core baluns are too bulky and narrowband.

The opening of the pyramidal shield is 150 x 150 mil at the feed terminals,
which are separated by 300 mil. Moreover, since two orthogonal non-planar
LP antennas surround the pyramidal shield for dual polarization,
the balanced ports of two baluns must be brought to the narrow tip of
the shield without inducing significant cross-polarization coupling. For this
reason, the shield volume must be partitioned by a metallic septum,
so that the baluns do not capacitively couple.

Figure 2 shows the tip of the dual polarization feed structures.
The shield volume is partitioned diagonally so that the balanced
port of each balun can be located to give maximum separation from
the surrounding grounded metal walls, which limits the highest
practical impedance on can achieve with a twin-lead type
transmission line. The balanced port of each balun must
must have the same impedance as the antenna terminals (240 ohms).
This can be conveniently achieved within the available space contraints
by fabricating offset 10 mil microstrip lines separated by 50 mil
on opposite sides of 15 mil thick Cuflon.
The balun leads are approximately 50 mils equidistant from the
septum and pyramidal
shield walls.

The twin-lead input of the balun must also be a 250/50 ohm transition
section. The input of the balun transducer section is a broadside-coupled
stripline pair; it is most conveniently mated to the 240 ohm twin
leads by asymmetrically tapering top and bottom conductors (fig 3a)
so that the outer diameter of the pair remain constant (60 mil).
The cross section
of our impedance matching twin-lead is shown in fig. 3b.
An IE3D simulation gives the impedance of this transmission line structure
as a function of a single linear parameter x (see fig 4). The 
impedance steps yield a 20-section 240/50 ohm transformer of minimum
length with a 
25 dB return loss (Klopfenstein taper). The number of steps limits
the upper frequency cutoff (> 10 GHz) and the total length determines
the low frequency cutoff (< 1 GHz). 

The transducer section consists of line conductor of nearly constant width 
(60 mil) broadside coupled through Cuflon substrate to a line which is
identical in width at the balanced port, then flares out exponentially (fig 5).
At the unbalanced port, the balun transducer is effectively a microstrip
line, where the bottom conductor becomes the ground plane (it is attached to
case ground along the end). The length
of the transducer and speed of the taper of the bottom conductor line
determine the passband of
the transducer section, which I optimized for 1 - 10 GHz.    

Test Results:

As shown in Fig 1 this taperline balun has two sections, the balanced
line portion which matches the antenna impedance to ~50 ohms and
a portion which actually performs the mode transduction.
One method to confirm the proper function of this balun design is
to connect two, end-to-end, at the balanced port (see fig. 6).
Clearly, the center of this circuit arrangement is a balanced twin lead
while the ends, which are microstrip-to-coax transitions, are
unbalanced. Figure 7 shows the transmission properties of this
circuit at room temperature are excellent: |S21| = 0.5 -- 0.8 dB
for freq. = 1 -- 1 GHz. Gating out reflections from the SMA
connectors, return loss from the transducer and impedance
matching sections is 20 -- 30 dB across the band of operation.
To check the phase performance of the balun circuit, a second
circuit is made by connecting two, end-to-end, as before, but with
the twin leads cross-connected. Since the balun transforms
a balanced signal in the E-plane to an unbalanced signal in the
H-plane, a comparison of our two back-to-back balun pairs should
exhibit a 180 degree phase difference over the passband. Figure 8
shows that this is indeed the case from 1 - 10 GHz. Phase errors
of only a few degrees occur over this frequency range.

Discussion:

The results previously quoted were for balun pairs mounted
in an open test fixture. For the balun to function properly, it must
work in a pyramidal shield. The effects of nearby conducting
walls are two-fold: they lower the characteristic impedance of both balanced
and unbalance transmission lines, making them more lossy;
they provide a conduction path 
for in-phase signals (the even mode) along twin leads.
At the tip of the pyramidal shield, the impedance of the balanced
transmission line mode (odd mode) and the even mode are 240 ohms
and 120 ohms respectively. Because the shield flares out at a 5 degree
half-angle, the impedance of the even mode continuously increases
from 120 ohms at the balanced port to 1K ohm at the unbalanced
port according to full wave simulations of the balun/shield combination,
hence the ratio of odd/even mode impedances is nearly 20:1. 
It is expected that with a good impedance match to the balun at
the unbalanced side the even mode excitation should be negligible.
In any case, this balun fabricated for Cuflon is certainly adequate for
range testing the ATA feed antenna patterns. Use of this balun
in a cryogenic front end requires scaling it to alumina, which means
reducing its length by more than a factor of 2, and making sure
significant excitation of the even mode does not occur, which can
raise the sidelobe patterns and make the beam profile asymmetric.
In table 1 I present the boundary design conditions and losses
if the alumina based circuit is cooled to 80 K.

One way to attenuate the even mode in an cryogenic balun is to attach 
a resistive card vane to the inside of the shield, perpendicular
to the cooled circuit board (see figure 9). Care must be taken not
to attenuate the odd mode in locations where the fringe fields
between balanced leads are high. To avoid this, th vane should probably 
have maximum extent at the balun "neck" where 60 mil lines are
broadside coupled and fringe fields are at minimum strength.  

