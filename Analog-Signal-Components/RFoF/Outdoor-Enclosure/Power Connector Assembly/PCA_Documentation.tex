\documentclass{article}
\usepackage[utf8]{inputenc}
%\usepackage{subcaption} %Captions for subfigure?
\paperheight=297mm % A4 paper dimensions
\paperwidth=210mm
\usepackage{float}

% Manually setting page dimensions
\setlength{\textheight}{235mm}
\setlength{\topmargin}{-1.2cm} 
\setlength{\textwidth}{15.5cm}
\setlength{\oddsidemargin}{0.5cm}
\setlength{\evensidemargin}{0.5cm}

% Include more figures type: eps,pdf,jpg, png
\usepackage{graphicx}
\usepackage[table,xcdraw]{xcolor}

% Hyperlinks:
\usepackage[hidelinks,colorlinks=false,bookmarks=false,linkcolor=black,urlcolor=black]{hyperref}

\pdfminorversion=5 
\pdfcompresslevel=9
\pdfobjcompresslevel=2\pdfobjcompresslevel=2


\begin{document}
\begin{center}
\textbf{\LARGE{IP 68 Outdoor Cable assembly for 120V Mains Supply}}\\
Alexander Pollak\\
November 19th, 2025\\
\end{center}


This document describes the assembly of the IP68 rated power cord that is used to supply mains voltage (120V AC). Figure \ref{fig:completed} shows a picture of the finished cord. Figure \ref{fig:WEIPU-3P} shows the wiring diagram for the WEIPU plug.
\\
\section{List of parts and tools needed:}

\begin{itemize}
\item WEIPU Plug Female 3P Solder UL (MPN\#: SP2110/S3II-1N(U))
\item Shrink Tube HS-TBG 6.4mm BK (Mouser\#: 650-TAT125014)
\item Shrink Tube HS-TBG 3.2mm BK (Mouser\#: 650-RNF-100-1/8-BK-S)
\item Power Cord various length (MPN\#: XXXXX)
\item Solder Iron, Side Cutter, Heat Gun.
\item \texttt{[ WEIPU Rcpt Male 3P Solder UL (MPN\#: SP2113/P3-1C(U)) ]}
\end{itemize}

\begin{figure}[H]
\includegraphics[width=\textwidth]{images/completed.jpeg}
\caption{Picture of the completed cable assembly.}
\label{fig:completed}
\end{figure}

\begin{figure}[H]
\includegraphics[width=\textwidth]{images/schematics/WEIPU-3P.png}
\caption{WEIPU 3 Pin connector wiring diagram. Note, this is standardized for all WEIPU 3P connectors and is only be used to supply mains voltage 120V AC to assemblies at HCRO. }
\label{fig:WEIPU-3P}
\end{figure}


\section{Manufacturing Instructions:}

The following section provides detailed instructions on how to prepare the power-cord assembly and the step-by-step process for assembling it.

\begin{figure}[H]
\includegraphics[width=\textwidth]{images/connector_prep.jpeg}
\caption{This picture shows the disassembled connector plug. Make sure that both parts on the left-had side are on the power cable before continuing with the assembly work. }
\label{fig:connector_prep}
\end{figure}

\begin{figure}[H]
\includegraphics[width=\textwidth]{images/cable_prep.jpeg}
\caption{First, prepare the cable by removing the outer jacked at a length of about 5\,cm. Also prepare the larger shrink tube (HS-TBG 6.4mm BK) by cutting a 5\,cm long section and putting it on the power cable.}
\label{fig:cable_prep}
\end{figure}

\begin{figure}[H]
\includegraphics[width=\textwidth]{images/shinktube_prep.jpeg}
\caption{Next prepare the smaller shrink tube (HS-TBG 3.2mm BK) and cut the individual vires to the length shown in the picture.}
\label{fig:shinktube_prep}
\end{figure}

\begin{figure}[H]
\includegraphics[width=\textwidth]{images/solder_1.jpeg}
\caption{Next, strip the wire, put solder on the wire. Put solder on the connector terminals. Put the shrink tube on the wire. Solder the wired onto the connector according to the wiring diagram in Figure \ref{fig:WEIPU-3P}. }
\label{fig:solder_1}
\end{figure}

\begin{figure}[H]
\includegraphics[width=\textwidth]{images/solder_2.jpeg}
\caption{Shows the finished soldering job. Next use the heat gun to put the shrink tube over the terminals and cable. }
\label{fig:solder_2}
\end{figure}


\end{document}
